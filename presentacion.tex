\documentclass{beamer}
%\usetheme{Berlin}
%\usetheme{Ilmenau}
%\usetheme{Dresden}
%\usetheme{Berkeley}
%\usetheme{Bergen}
%\usetheme{Boadilla}
%\usetheme{Copenhagen}
%\usetheme{Hannover}
%\usetheme{Luebeck}
%\usetheme{AnnArbor}
%\usetheme{Darmstadt}
%\usetheme{Frankfurt}
\usetheme{Madrid}%azulito-li;la
%\usetheme{Warsaw}%int
%\usetheme{Antibes}
%\usetheme{CambridgeUS}%rojo-gris
%\usetheme{Malmoe}
%\usetheme{PaloAlto}

\usepackage{graphicx}
\usepackage{adjustbox}
\usepackage[utf8]{inputenc}
\usepackage[spanish]{babel}
\usepackage{url}
\usepackage{multirow}
\usepackage{booktabs}
%\usepackage{beamerthemeshadow}
\usepackage{caption}


\hyphenation{}


%Arreglos en el footnote
\makeatletter
\setbeamertemplate{footline}
{
  \leavevmode%
  \hbox{%
  \begin{beamercolorbox}[wd=.333333\paperwidth,ht=2.25ex,dp=1ex,center]{author in head/foot}%
    \usebeamerfont{author in head/foot}\insertshortauthor
  \end{beamercolorbox}%
  \begin{beamercolorbox}[wd=.58\paperwidth,ht=2.25ex,dp=1ex,center]{title in head/foot}%
    \usebeamerfont{title in head/foot}\inserttitle
  \end{beamercolorbox}%
  \begin{beamercolorbox}[wd=.1\paperwidth,ht=2.25ex,dp=1ex,center]{date in head/foot}%
    \usebeamerfont{date in head/foot}\insertframenumber{} / \inserttotalframenumber\hspace*{2ex} 
  \end{beamercolorbox}}%
  \vskip0pt%
}
\makeatother
%APAGAR SOMBRAS EN BLOQUES
%\setbeamertemplate{blocks}[rounded][shadow=false]

%Apagar el menu de presentacion
\setbeamertemplate{navigation symbols}{}

\begin{document}
\title{Optimización de viajes compartidos en taxis utilizando algoritmos evolutivos}
\author[G. Fagúndez \and R. Massobrio]{Gabriel Fagúndez de los Reyes \and Renzo Massobrio} 
\institute[]{Facultad de Ingeniería,\\
Universidad de la República,\\
Montevideo, Uruguay}

\pgfdeclareimage[height=1.7cm]{university-logo}{logo}
\titlegraphic{\pgfuseimage{university-logo}}
\date{}

\frame{\frametitle{}\titlepage} 
\frame{\frametitle{Contenido}\tableofcontents} 

% Pausas transparentes
\setbeamercovered{transparent}




% Slide - Introducción ==============================================================
\section{Introducción}
\frame{\tableofcontents[currentsection]}

\frame{
	\frametitle{Motivación} 
	\begin{block}{Los viajes compartidos (\textit{Car Pooling})}
		\begin{itemize}
			\item Beneficios ecológicos y económicos, individuales y colectivos.
			\item Iniciativas:
			\begin{itemize}
			\item Carriles exclusivos
			\item Campañas para compartir los viajes al trabajo
			\item Aplicaciones para encontrar compañeros de viaje
			\end{itemize}
		\end{itemize}
	\end{block}
	
	\pause 
	
	\begin{block}{Los viajes compartidos en taxis (\textit{Taxi Pooling})}
		\begin{itemize}
			\item Taxis, medio de transporte rápido y confiable.
			\item Raramente viajan a capacidad completa.
			\item Tarifas altas desalientan a los usuarios.
			\item Impactan en la congestión del tráfico y en la contaminación.
		\end{itemize}
	\end{block}
}




% Slide - Definición del problema ==============================================================
\section{Definición del problema} 
\frame{\tableofcontents[currentsection]}

\frame{
	\frametitle{Descripción del problema}
	\begin{block}{Problema de viajes compartidos en taxis (PVCT)}
		Un grupo de personas en un \textbf{mismo lugar de origen}, desean viajar hacia \textbf{diferentes destinos} utilizando taxis de forma compartida. Hallar la cantidad de taxis y la asignación de pasajeros para minimizar el \alert{costo total}.

		%Se busca determinar la cantidad de taxis, la asignación de pasajeros y las rutas a seguir, de forma de \alert{minimizar el costo total del grupo de pasajeros}.
	\end{block}
	
	\pause
	
	\begin{block}{Consideraciones}
		\begin{itemize}
			\item Cada taxi puede trasladar a un número limitado de pasajeros.
			\item El número máximo de taxis para $N$ pasajeros es $N$.
			\item Costo de un taxi = \textbf{costo inicial} $+$ \textbf{costo por trayectos}.
			\item No se consideran otros costos (e.g. esperas, propinas, peajes).
		\end{itemize}		
	\end{block}
}

\frame{
	\frametitle{Formulación del problema}
	\begin{itemize}
		\item Un conjunto de pasajeros $P$ que viajan desde un origen $O$ a un conjunto de destinos $D$.

		\item Un conjunto de taxis $T$ y una función $C:T\rightarrow\{0,1,\ldots,C_{MAX}\}$ con $C_{MAX}$ máxima capacidad permitida en un taxi.

		\item Una constante $B$ que determina el costo inicial del taxi.
		
		\item Una función de distancia, $dist: \lbrace \lbrace O \rbrace \cup D \rbrace \times D \rightarrow \mathbb{R}^+_0$, y una función de costo asociado a la distancia recorrida por cada taxi, $cost: \mathbb{R}^+_0 \rightarrow \mathbb{R}^+_0$.
		
	\end{itemize}\pause

	\centering
	\textbf{Se busca una planificación $f:P \rightarrow T \times \lbrace 1, \ldots,C_{MAX} \rbrace$ que \alert{minimice la función de costo total ($CT$)}.}
	\begin{equation*}
			CT  =  \sum\limits_{t_{i}, C(t_{i})\neq0} \Bigg[B+\sum\limits_{j=1}^{C(t_{i})}cost\bigg(dist \underbrace{\Big(dest\big(f^{-1}(t_{i},j-1)\big),dest\big(f^{-1}(t_{i},j)\big)\Big)}_{\text{destinos consecutivos en el recorrido del taxi } t_i}\bigg)\Bigg]
	\end{equation*}
}

\frame{
	\frametitle{Variante multiobjetivo del PVCT: formulación matemática}
	\pause
		\begin{itemize}
		\item Se busca minimizar el \alert{costo total} y la \alert{demora total}.\pause
		\item Se incorporan dos variables:
		\begin{itemize}
			\item Se consideran vehículos con \alert{diferentes capacidades}.
			\item Cada pasajero tiene un \alert{nivel de apuro} asociado.
		\end{itemize}
	\end{itemize}\pause

	\begin{equation*}
			CT  =  \sum\limits_{t_{i}, C(t_{i})\neq0} \Bigg[B+\sum\limits_{j=1}^{C(t_{i})}cost\bigg(dist\overbrace{\Big(dest\big(f^{-1}(t_{i},j-1)\big),dest\big(f^{-1}(t_{i},j)\big)\Big)}^{\text{destinos consecutivos en el recorrido del taxi } t_i}\bigg)\Bigg]
	\end{equation*}\pause
	\begin{equation*}
		\begin{split}
			DT = \sum\limits_{t_{i}} \Bigg[\sum\limits_{j=1}^{C(t_{i})}\bigg[ &  \overbrace{\sum\limits_{h=1}^{j} time\Big(dest\big(f^{-1}(t_{i},h-1)\big), dest\big(f^{-1}(t_{i},h)\big)\Big)}^{\text{tiempo efectivo de traslado del pasajero en la posición } j \text{ del taxi } t_i} \\
			& - \bigg( \underbrace{tol\big(f^{-1}(t_{i},j)\big) + time\Big(O, dest\big(f^{-1}(t_{i},j))\Big)\bigg)}_{\text{tiempo tolerado por el pasajero en la posición } j \text{ del taxi } t_i}\bigg]\Bigg]
		\end{split}	
	\end{equation*}
}

\frame{
	\frametitle{Complejidad del PVCT}
	\begin{block}{Complejidad}
	%El PVCT tiene varios puntos en común con dos conocidos problemas: el \textit{Car Pooling Problem (CPP)} y el \textit{Vehicle Routing Problem (VRP)}.

	\begin{itemize}
	\item Baldacci et al. (2004):
	\begin{itemize}
		\item Variante del \textit{Car Pooling Problem (CPP)}.
		\item Solución al problema de compartir vehículos en el trabajo.
	\end{itemize}
	
	\item Caso especial del \textit{Vehicle Routing Problem (VRP)} con demanda unitaria:
	\begin{itemize}
		\item Letcheford et al. (2002) demostró que es $\mathcal{NP}$--difícil.
	\end{itemize}
	\item Similitudes determinan que nuestro problema también es $\mathcal{NP}$--difícil.
	\end{itemize}
	\end{block}

	\pause

	\begin{block}{Estrategias de resolución}
	\begin{itemize}
		\item Con instancias de tamaños realistas los algoritmos exactos tradicionales no resultan útiles para una planificación eficiente.
		\item \alert{Heurísticas} y \alert{metaheurísticas} son necesarias para calcular soluciones de calidad aceptable en tiempos razonables.
	\end{itemize}
	\end{block}

}



% Slide - Trabajo relacionado ==============================================================
\section{Trabajos relacionados} 
\frame{\tableofcontents[currentsection]}

\frame{
	\frametitle{Trabajo relacionado}

	\begin{block}{Car pooling problem (CPP)}
	Yan et al. (2011): \textbf{CPP con histórico de viajes} (relajación lagrangeana). 	\end{block}
	\pause
	\begin{block}{Dial--a--ride problem (DARP)}
	Cordeau et al. (2003): \textbf{DARP estático con ventanas de tiempo}. \\\hspace{.8cm} {\small Búsqueda tabú con tiempos de ejecución de hasta 90 minutos.}
	\end{block}


	\pause
	\begin{block}{Taxi pooling problem (TPP)}
	Tao et al. (2007): heurísticas ávidas para \textbf{one--to--many} y many--to--one. \\ \hspace{.8cm}{\small Las mejoras se reportan en términos absolutos.}

	Ma et al. (2013): TPP dinámico con pedidos en tiempo real.\\ \hspace{.8cm}{\small 13\% de ahorro en distancia con un algoritmo ávido en \textbf{instancias realistas}.}
	\end{block}

	\pause
	\begin{block}{Resumen}
	Pocos trabajos \alert{centrados en el usuario} y con un enfoque \alert{multiobjetivo}.
	\end{block}
}





% Slide - Implementación ==============================================================
\section{Implementación} 
\frame{\tableofcontents[currentsection]}

\frame{
	\frametitle{Algoritmos evolutivos}
	\begin{block}{Definición}
		\begin{itemize}
			\item Técnicas estocásticas que emulan el proceso de evolución natural de las especies.\pause
			\item Aplicadas a problemas de \alert{optimización}, búsqueda y aprendizaje.\pause
			\item Técnica iterativa (\textbf{generación}) que aplica operadores estocásticos sobre un conjunto de individuos (\textbf{población}).\pause
			\item Cada individuo codifica una solución tentativa al problema y tiene un valor de \textbf{fitness}.\pause
			\item El propósito es mejorar el fitness de los individuos en la población mediante la aplicación de \textbf{operadores evolutivos}.\pause
			\item Los operadores guian al algoritmo evolutivo hacia soluciones tentativas de mayor calidad.
		\end{itemize}
	\end{block}
}


\frame{
	\frametitle{AE para el PVCT monoobjetivo}
	\begin{columns}[totalwidth=\linewidth]
	\column{0.43\linewidth}
		\begin{block}{Aspectos comunes}
		\begin{itemize}
		\item Tuplas de largo $2N-1$\\
		\hspace{.5cm}{\small$N=$ \#pasajeros.}
		\pause
		
		\item Inicialización:\\
		\hspace{.5cm}{\small aleatoria y ávida.}
		\pause
		
		\item Cruzamiento basado en posición (PBX).% + función correctiva.

		\pause
		\item Mutación por intercambio (EM).% + función correctiva.
		\pause


		\item Función correctiva:\\ \hspace{.5cm}{\small desplaza ceros para}\\ \hspace{.5cm}{\small romper secuencias de}\\ \hspace{.5cm}{\small dígitos inválidas.}
		\pause

		\item Implementados en Malva.
		\end{itemize}
		\end{block}

	\column{0.57\linewidth}
		\centering

		\includegraphics[width=0.9\textwidth]{distribucion.png}

		\only<1-2>{\mbox{\phantom{\includegraphics[width=0.9\textwidth]{pbx_simple.png}}}}
		\includegraphics<3->[width=0.9\textwidth]{pbx_simple.png}

		\only<1-3>{\mbox{\phantom{\includegraphics[width=0.9\textwidth]{em.png}}}}%
		\includegraphics<4->[width=0.9\textwidth]{em.png}

		\only<1-4>{\mbox{\phantom{\includegraphics[width=0.9\textwidth]{temporal.png}}}}%
		\includegraphics<5->[width=0.9\textwidth]{temporal.png}
	\end{columns}

}

\frame{
	\frametitle{AE para el PVCT monoobjetivo}
	\begin{block}{AE secuencial \textit{(seqEA)}}
		\begin{itemize}
			\item Selección proporcional.
		\end{itemize}
	\end{block}
	\pause

	\begin{block}{Micro AE paralelo (\textit{p$\mu$EA})}
		\begin{itemize}
			\item Se busca \textbf{mejorar el desempe\~no} mediante el paralelismo.\pause
			\item \alert{Modelo de subpoblaciones distribuidas}: divide la poblaci\'on en \textbf{islas} que intercambian individuos mediante \textbf{migración}.\pause
			
			\begin{centering}
				\centering
				\only<1-2>{\mbox{\phantom{\includegraphics[width=0.5\textwidth]{migracion.png}}}}
				\hspace{3cm} \includegraphics<3->[width=0.5\textwidth]{migracion.png}
			\end{centering}
			
			\item Poblaciones pequeñas.\pause
			\item Selección por torneo $(m,k)$.\pause
			\item Migración asíncrona, con topología de anillo unidireccional.

		\end{itemize}
	\end{block}
}


\frame{
	\frametitle{AE para el PVCT multiobjetivo}
	\begin{block}{Aspectos comunes}
		\begin{itemize}
			\item MOEA: Acercarse al frente de Pareto del problema (\textbf{convergencia}) y muestrear adecuadamente el frente de soluciones (\textbf{diversidad}).\pause
			\item Función correctiva considera vehículos de distintas capacidades.\pause
			\item Inicialización de la población ávida y selección por torneo.
		\end{itemize}
	\end{block}


	\pause
		\begin{block}{MOEA paralelo con micro--población y descomposición de dominio (\textit{p$\mu$MOEA/D})}

		\begin{columns}[totalwidth=\textwidth]
		\column{0.50\textwidth}
			Agregación lineal de los objetivos:
			\\\hspace{1.5cm}$F = w_C\times CT + w_D \times DT$, \small \\\hspace{2.3cm}{\tiny$w_C = [0:\frac{1}{\#islas}:1]$, $w_D = 1 - w_C$}.
		\column{0.40\textwidth}
			\centering
			\only<1-3>{\mbox{\phantom{\includegraphics[width=1\textwidth]{pmuMOEAD.png}}}}
			\includegraphics<4->[width=1\textwidth]{pmuMOEAD.png}
		\end{columns}
		\end{block}
	\pause

	\begin{block}{MOEA explícito (\textit{NSGA--II})}
		Ordenamiento no--dominado (elitista), \textit{crowding} para preservar diversidad.
	\end{block}
}

% Slide - Evaluación Experimental ==============================================================
\section{Evaluación experimental} 
\frame{\tableofcontents[currentsection]}

\frame{
	\frametitle{Generación de instancias}

	\begin{block}{Generación de puntos realistas en el mapa}
		\begin{itemize}
		\item Generador de Pedidos de Taxis (TQG) con datos de GPS de taxis de Beijing (Ma et al., 2013).\pause
		\item Script para obtener instancias de un origen a muchos destinos.\pause
		\item API para obtener tarifas TaxiFareFinder (TFF).\pause
		\item Instancias en Montevideo generadas manualmente.
		\end{itemize}
	\end{block}
	\pause
	\begin{block}{Instancias generadas}

	\begin{itemize}\setlength\itemsep{1pt}
		\item \textbf{6 chicas:} 10 y 15 pasajeros (Beijing).
		\item \textbf{6 medianas:} 15 y 25 pasajeros (Beijing).
		\item \textbf{6 grandes:} 25 y 45 pasajeros (Beijing).
		\item \textbf{4 en Montevideo:} 8 y 17 pasajeros (Montevideo).
	\end{itemize}
	
	\alert{22} instancias para el PVCT monoobjetivo y \alert{88} instancias para el PVCT multiobjetivo, variando capacidades y tolerancias.
	
	\end{block}

}

\frame{
	\frametitle{Metodología}
	\begin{columns}[totalwidth=\textwidth]
	\column{0.5\textwidth}
	\begin{block}{Entorno de ejecución}
		\begin{itemize}
		\item Evaluación experimental realizada en el Cluster FING.
		\item Sin compartir recursos para evitar interferencias.
		\end{itemize}
	\end{block}
	\column{0.5\textwidth}
	\centering
		\includegraphics[width=.85\textwidth]{cluster}
	\end{columns}
	\pause
	\begin{block}{Ejecuciones}
	\begin{itemize}\setlength\itemsep{1pt}
		\item \alert{30 ejecuciones independientes} de cada algoritmo sobre cada instancia.
		\item Criterio de parada: \textbf{10.000 generaciones} (planificación en línea).
	\end{itemize}
	\end{block}\pause
	\begin{block}{Comparación de resultados}
	\begin{itemize}
		\item Tests estadísticos sobre las distribuciones de resultados:
		\begin{itemize}
		\item Shapiro--Wilk sobre cada muestra para contrastar normalidad.
		\item Kruskal--Wallis para comparar las muestras entre sí.
		\end{itemize}
		\item En ambos tests se utiliza un nivel de confianza del 95\% ($\alpha=0.05$).
	\end{itemize}	
	\end{block}

}

\frame{
	\frametitle{PVCT monoobjetivo}
	% \begin{block}{Entorno de ejecución}
	% 	\begin{itemize}
	% 	%\item La evaluación experimental fue realizada en el Cluster FING.
	% 	\item \textbf{seqEA}: Dell Power Edge 2950, \textbf{1 núcleo} de Intel Xeon E5430 2.66GHz, 8GB RAM.
	% 	\item \textbf{p$\mu$EA}: HP Proliant DL585, \textbf{24 núcleos} de AMD Opteron 2.09GHz, 48GB RAM.
	% 	\end{itemize}
	% \end{block}\pause
	\begin{block}{Configuración paramétrica}
	\begin{itemize}
		\item \textbf{seqEA}: %20 ejecuciones de 2000 generaciones sobre 3 instancias.\\
		{\small $\text{\#\textit{P}} \in \lbrace 150; 200; \alert{250}\rbrace$; $p_C \in \lbrace \alert{0.6}; 0.75; 0.95\rbrace$; $p_M \in \lbrace 0.001; 0.01; \alert{0.1}\rbrace$.}\pause	
		\item \textbf{p$\mu$EA}: micro--población de 15 individuos, torneo ($m=2$, $k=1$), migración cada 500 generaciones. \\%20 ejecuciones de 100.000 generaciones sobre 5 instancias.\\
		{\small $p_C \in \lbrace 0.6; \alert{0.75}; 0.95\rbrace$; $p_M \in \lbrace 0.001; 0.01; \alert{0.1}\rbrace$.}
	\end{itemize}
	\end{block}

}

\frame{	
	\frametitle{Algoritmo ávido}
	Toma \textbf{decisiones localmente óptimas} y emula el comportamiento de un grupo de usuarios humanos. Utiliza ideas de los trabajos relacionados. 
	\includegraphics[width=1.\linewidth]<1>{greedy_costo_1}
	\includegraphics[width=1.\linewidth]<2>{greedy_costo_2}
	\includegraphics[width=1.\linewidth]<3>{greedy_costo_3}
	\includegraphics[width=1.\linewidth]<4>{greedy_costo_4}
	\includegraphics[width=1.\linewidth]<5>{greedy_costo_5}
	\includegraphics[width=1.\linewidth]<6>{greedy_costo_6}
	\includegraphics[width=1.\linewidth]<7>{greedy_costo_7}
	\includegraphics[width=1.\linewidth]<8>{greedy_costo_8}

}

\frame{
	\frametitle{Comparativa de métodos de inicialización}

	\begin{block}{Resultados \textit{seqEA}}
	\begin{itemize}
		\item Inicialización ávida supera a inicialización aleatoria en \alert{10} instancias.
		\item Inicialización aleatoria supera a inicialización ávida en \alert{2} instancias.
		\item No hay diferencias estadísticamente significativas en 10 instancias.
	\end{itemize}
	\end{block}
	\pause
	\begin{block}{Resultados \textit{p$\mu$EA}}
	\begin{itemize}
		\item Inicialización ávida supera a inicialización aleatoria en \alert{11} instancias.
		\item No hay instancias en las que la inicialización aleatoria supere a la inicialización ávida.
		\item No hay diferencias estadísticamente significativas en 11 instancias.
	\end{itemize}
	\end{block}
	\pause
	\begin{block}{Conclusión}
	Se utiliza la \alert{inicialización ávida} para el resto de la evaluación experimental.
	\end{block}
}

\frame{
	\frametitle{Mejoras \textit{seqEA} sobre algoritmo ávido}
	\centering
\includegraphics[width=9cm,height=7cm]{./evaluacion_experimental/mejoras_greedy_clei/greedy_costo_Chicas}
	
	Mejoras en \textbf{todas} las instancias sobre el algoritmo ávido (hasta \alert{35.9\%}).

	\small{\textit{``Online taxi sharing optimization using evolutionary algorithms, CLEI, 2014''}}

}

\frame{
	\frametitle{Mejoras \textit{p$\mu$EA} sobre algoritmo ávido}
	\begin{columns}[t]
\column{.5\textwidth}
\centering
\includegraphics[width=5cm,height=3.5cm]{./evaluacion_experimental/mejoras_greedy_alio/greedy_costo_Chicas}\\
\includegraphics[width=5cm,height=3.5cm]{./evaluacion_experimental/mejoras_greedy_alio/greedy_costo_Medianas}
\column{.5\textwidth}
\centering
\includegraphics[width=5cm,height=3.5cm]{./evaluacion_experimental/mejoras_greedy_alio/greedy_costo_Grandes}\\
\includegraphics[width=5cm,height=3.5cm]{./evaluacion_experimental/mejoras_greedy_alio/greedy_costo_Montevideo}
\end{columns}
	
	Mejoras en \textbf{todas} las instancias sobre el algoritmo ávido (hasta \alert{41.0\%}).

	\small{\textit{``A parallel micro evolutionary algorithm for taxi sharing optimization, ALIO''}}
}

\frame{
	\frametitle{Comparativa \textit{seqEA} vs. \textit{p$\mu$EA}}
	\centering
\begin{adjustbox}{max width=0.55\textwidth}
\begin{tabular}{ccrrrrr}
\multicolumn {2}  {c} {\multirow{2}{*}{\textit{instancia}}}  & \multicolumn {2}  {c} {\textit{seqEA}} &\multicolumn {2}  {c}{\textit{p$\mu$EA}} &\multirow {2}  {*}{\textit{pvK--W}}\\
	   & 	&\multicolumn {1}  {c}{$min(c)$}	& \multicolumn {1}  {c}{$\overline{c}\pm std$}	& \multicolumn {1}  {c}{$min(c)$} &\multicolumn {1}  {c}{$\overline{c}\pm std$}	& \\
\midrule
\multirow {6}  {*} {chicas}     	& \#1             	& 164.4  				& 165.6$\pm$2.0                		& 164.4 & \alert{164.4}$\alert{\pm}$\alert{0.0}        & 0.2$\times 10^{-3}$   \\
                                		& \#2             	& 220.7  				& 225.7$\pm$5.0                		& 220.7 & \alert{220.7}$\alert{\pm}$\alert{0.0}        & 9.7$\times 10^{-6}$   \\
                                		& \#3             	& 160.4  				& 160.4$\pm$0.0                		& 160.4 & 160.4$\pm$0.0         & 1.0                 \\
                                		& \#4             	& 181.3  				& 181.3$\pm$0.1       			& 181.3 & 182.4$\pm$1.9       & 0.5$\times 10^{-1}$     \\
                                		& \#5             	& 152.1  				& 155.6$\pm$4.5                		& 152.1 & \alert{152.1}$\alert{\pm}$\alert{0.0}        & 5.1$\times 10^{-6}$   \\
                                		& \#6             	& 118.4  				& 119.6$\pm$2.5                		& 118.4 & \alert{118.4}$\alert{\pm}$\alert{0.0}        & 0.1$\times 10^{-1}$     \\
\hline
\multirow {6}  {*} {medianas}   	& \#1             	& 211.9 				& 216.0$\pm$4.2                 		& 211.9 & \alert{211.9}$\alert{\pm}$\alert{0.0}        & 5.2$\times 10^{-11}$   \\
                                		& \#2             	& 428.6 				& 444.1$\pm$11.7                		& 427.9 & \alert{429.4}$\alert{\pm}$\alert{1.6}        & 7.0$\times 10^{-10}$  \\
                               		 	& \#3             	& 361.7 				& 378.7$\pm$6.5                 		& 364.5 & \alert{370.4}$\alert{\pm}$\alert{4.5}        & 1.6$\times 10^{-6}$   \\
                                		& \#4             	& 267.5 				& 279.8$\pm$5.5                 		& 266.8 & \alert{266.8}$\alert{\pm}$\alert{0.0}        & 7.6$\times 10^{-12}$   \\
                               	 		& \#5             	& 479.3 				& 487.1$\pm$6.5                 		& 479.6 & \alert{479.8}$\alert{\pm}$\alert{0.2}        & 5.1$\times 10^{-7}$   \\
                                		& \#6             	& 306.0 				& 321.2$\pm$7.7                 		& 306.0 & \alert{307.7}$\alert{\pm}$\alert{3.4}        & 2.0$\times 10^{-9}$   \\
\hline
\multirow {6}  {*} {grandes}    	& \#1             	& 421.9 				& \alert{435.1}$\alert{\pm}$\alert{5.0}        		& 425.9 & 437.7$\pm$3.2       & 0.1$\times 10^{-1}$     \\
                                		& \#2             	& 479.3 				& 489.9$\pm$4.3                 		& 477.0 & \alert{481.1}$\alert{\pm}$\alert{2.3}        & 1.9$\times 10^{-9}$   \\
                                		& \#3             	& 332.8 				& 349.7$\pm$7.7                 		& 326.3 & \alert{331.7}$\alert{\pm}$\alert{4.0}        & 2.6$\times 10^{-10}$  \\
                                		& \#4             	& 351.1 				& 390.7$\pm$26.3                		& 338.4 & \alert{344.8}$\alert{\pm}$\alert{6.1}        & 5.1$\times 10^{-11}$   \\
                                		& \#5             	& 395.9 				& 429.6$\pm$16.2                		& 370.2 & \alert{380.0}$\alert{\pm}$\alert{4.4}        & 2.7$\times 10^{-11}$   \\
                                		& \#6             	& 360.8 				& 382.4$\pm$8.1                 		& 343.8 & \alert{350.6}$\alert{\pm}$\alert{3.8}        & 2.6$\times 10^{-11}$   \\
\hline
\multirow {4}  {*} {Montevideo} & \#1            	& 168.4 				& 168.4$\pm$0.0                 		& 168.4 & 168.4$\pm$0.0         & 1.0                 \\
                                		& \#2            	& 319.3 				& 331.2$\pm$3.8                 		& 324.9 & \alert{328.6}$\alert{\pm}$\alert{3.2}        & 5.6$\times 10^{-6}$   \\
                                		& \#3             	& 266.7 				& 269.1$\pm$2.3                 		& 266.7 & \alert{266.7}$\alert{\pm}$\alert{0.0}        & 3.1$\times 10^{-7}$   \\
                                		& \#4             	& 303.2 				& 304.7$\pm$0.5                 		& 304.1 & 304.5$\pm$0.4         & 0.1       \\\bottomrule
\end{tabular}
\end{adjustbox}

\vspace{0.15cm}
	\textit{p$\mu$EA} supera a \textit{seqEA} en \alert{17 de 22} instancas. Únicamente en \alert{1} instancia \textit{seqEA} supera a \textit{p$\mu$EA}.


}

\frame{
	\frametitle{Evolución del costo a lo largo de una ejecución}	
	\begin{columns}[t]
\column{.5\textwidth}
\centering
\includegraphics[width=4.5cm,height=3.5cm]{./evaluacion_experimental/fitness_sobre_tiempo/chicas_1.pdf}\\
\includegraphics[width=4.5cm,height=3.5cm]{./evaluacion_experimental/fitness_sobre_tiempo/medianas_2.pdf}
\column{.5\textwidth}
\centering
\includegraphics[width=4.5cm,height=3.5cm]{./evaluacion_experimental/fitness_sobre_tiempo/grandes_2.pdf}\\
\includegraphics[width=4.5cm,height=3.5cm]{./evaluacion_experimental/fitness_sobre_tiempo/montevideo_1.pdf}
\end{columns}

	\textit{p$\mu$EA} alcanza mejores soluciones que \textit{seqEA} en menos tiempo. 

	En el mejor caso alcanza una \alert{aceleración de $7.5$x} ($4.6$x en promedio).

}

\frame{
	\frametitle{PVCT multiobjetivo}

	% \begin{block}{Entorno de ejecución}
	% 	\begin{itemize}
	% 		%\item La evaluación experimental fue realizada en Cluster FING.
	% 		\item \textbf{p$\mu$MOEA/D}: HP Proliant DL585, \textbf{24 núcleos} de AMD Opteron 6272 2.09GHz, 48GB RAM.
	% 		\item \textbf{NSGA--II}: HP Proliant DL385 G7, \textbf{1 núcleo} de AMD Opteron 6172 2.10GHz, 72GB RAM.
	% 	\end{itemize}
	% \end{block}\pause
	
	\begin{block}{Configuración paramétrica}
		\begin{itemize}
			\item \textbf{p$\mu$MOEA/D}: %30 ejecuciones de 20000 generaciones sobre 4 instancias. \\
			{\small $\#P=15$; selección por torneo ($m=2,k=1$); migración cada 1000 generaciones reemplazando a los peores individuos\\
			$p_C \in \lbrace 0.6; \alert{0.75}; 0.95\rbrace$; $p_M \in \lbrace 0.001; 0.01; \alert{0.1}\rbrace$}\\\pause
			\item \textbf{NSGA--II}: %30 ejecuciones de 5000 generaciones sobre 4 instancias. \\
			{\small $\#P=80$; selección por torneo ($m=2,k=1$);\\
			$p_C \in \lbrace 0.6; \alert{0.75}; 0.95\rbrace$; $p_M \in \lbrace 0.001; 0.01; \alert{0.1}\rbrace$.}\\
		\end{itemize}
	\end{block}
}

\frame{
	\frametitle{Algoritmos ávidos}
	
	\begin{block}{Algoritmo ávido para minimizar el costo}
		Similar al de la variante monoobjetivo pero considerando las distintas capacidades de los vehículos.
	\end{block}\pause
	\begin{block}{Algoritmo ávido para minimizar la demora}
		\begin{itemize}
		\item Se crea un taxi vacío para cada pasajero con nivel máximo de apuro y se los ubica en la primera posición. 
		%\item Se ubica a cada pasajero con mayor nivel de apuro en la primera posición del taxi.
		\item Luego, se recorre la lista de pasajeros no asignados en orden de apuro, colocándolos en el taxi que minimice su demora.
		\item Si el taxi alcanza la máxima capacidad disponible, se lo considera \textit{completo} y no acepta más pasajeros.
		\end{itemize}
	\end{block}
}

\frame{
	\frametitle{Resultados numéricos}
		
	\begin{block}{\scriptsize{\textit{p$\mu$MOEA/D}:} \tiny{``Planificación multiobjetivo de viajes compartidos en taxis utilizando un micro algoritmo evolutivo paralelo'' (MAEB)}}
	Hasta \alert{101.2\%} de mejora en demora y \alert{72.8\%} en costo sobre ávidos.\pause
		
		\centering
\begin{adjustbox}{max width=1\textwidth}
\begin{footnotesize}
\begin{tabular}{lrrrrr}
\toprule
			&\multicolumn{1}{c}{\#ND}& \multicolumn{1}{c}{DG} & \multicolumn{1}{c}{spacing} & \multicolumn{1}{c}{spread} & \multicolumn{1}{c}{RHV} \\\midrule
 chicas     & 8.5$\pm$2.1 (16.0) & 3.1$\pm$2.5 (0.0) & 740.2$\pm$746.3 (58.1)    & 0.6$\pm$0.2 (0.1) & 0.9$\pm$0.1 (1.0) \\
 medianas   & 9.1$\pm$2.2 (19.0) & 5.7$\pm$2.5 (0.0) & 1448.5$\pm$1064.1 (141.6) & 0.6$\pm$0.1 (0.1) & 0.9$\pm$0.1 (1.0) \\
 grandes   & 8.5$\pm$2.2 (17.0) & 7.9$\pm$3.4 (2.0) & 2917.2$\pm$2041.5 (175.3) & 0.6$\pm$0.1 (0.0) & 0.8$\pm$0.1 (1.0) \\
 Montevideo & 8.0$\pm$2.1 (14.0) & 3.0$\pm$2.0 (0.0) & 663.5$\pm$542.4 (61.5)    & 0.6$\pm$0.2 (0.0) & 0.9$\pm$0.0 (1.0) \\
\bottomrule
\end{tabular}
\end{footnotesize}
\end{adjustbox}
		\small{Buena convergencia y diversidad. Pocas soluciones no dominadas.}
	\end{block}
	\pause
	\begin{block}{\textit{NSGA--II:}}
		Hasta \alert{105.2\%} de mejora en demora y \alert{75.1\%} en costo sobre ávidos.\pause
		
		\begin{adjustbox}{max width=1\textwidth}
\centering
\begin{footnotesize}
\begin{tabular}{lrrrrr}
\toprule
			&\multicolumn{1}{c}{\#ND}& \multicolumn{1}{c}{DG} & \multicolumn{1}{c}{spacing} & \multicolumn{1}{c}{spread} & \multicolumn{1}{c}{RHV} \\\midrule

 chicas     & 32.6$\pm$9.5 (55.0)  & 0.3$\pm$0.6 (0.0) & 236.2$\pm$222.7 (43.2) & 0.9$\pm$0.1 (0.7) & 1.0$\pm$0.0 (1.0) \\
 medianas   & 54.5$\pm$4.2 (67.0)  & 1.0$\pm$0.7 (0.0) & 193.6$\pm$202.4 (26.2) & 0.7$\pm$0.2 (0.4) & 1.0$\pm$0.0 (1.0) \\
 grandes    & 55.2$\pm$3.5 (67.0)  & 1.8$\pm$1.1 (0.4) & 243.6$\pm$229.8 (26.4) & 0.7$\pm$0.2 (0.4) & 1.0$\pm$0.0 (1.0) \\
 Montevideo & 43.9$\pm$16.4 (61.0) & 0.4$\pm$0.5 (0.0) & 142.3$\pm$143.2 (20.8) & 0.8$\pm$0.1 (0.5) & 1.0$\pm$0.0 (1.0) \\

\bottomrule
\end{tabular}
\end{footnotesize}
\end{adjustbox}
		\small{Mayor cantidad de puntos no dominados (hasta 67/80). Buena convergencia y diversidad en las soluciones encontradas.}

	\end{block}	
	
}


\frame{
	\frametitle{Frentes de Pareto: p$\mu$MOEA/D vs. NSGA--II}
	\begin{columns}[t]
\column{.5\textwidth}
\centering
\includegraphics[width=4.5cm,height=3.5cm]{./evaluacion_experimental/fp_comparacion/Ch_6_1.png}\\
\includegraphics[width=4.5cm,height=3.5cm]{./evaluacion_experimental/fp_comparacion/Me_5_3.png}
\column{.5\textwidth}
\centering
\includegraphics[width=4.5cm,height=3.5cm]{./evaluacion_experimental/fp_comparacion/Gr_5_2.png}\\
\includegraphics[width=4.5cm,height=3.5cm]{./evaluacion_experimental/fp_comparacion/Mo_4_3.png}
\end{columns}

	\alert{\textit{NSGA--II} alcanza mejores soluciones}: mayor cantidad de puntos no dominados distribuidos homogéneamente a lo largo del frente.

	
}

\frame{
	\frametitle{Mejora frente a algoritmos ávidos vs. tiempo de ejecución}
	%El enfoque multiobjetivo explícito de \textit{NSGA--II} permite alcanzar mejores resultados que el enfoque por descomposición de dominio aplicado en \textit{p$\mu$MOEA/D}. Sin embargo, los tiempos de ejecución son significativamente mayores.

	\begin{columns}[t]
	\column{.5\textwidth}
	\centering
		\includegraphics[width=6.075cm,height=4.725cm]{./evaluacion_experimental/tradeoff_maeb_mic/costo}\\
	\column{.5\textwidth}
	\centering
		\includegraphics[width=6.075cm,height=4.725cm]{./evaluacion_experimental/tradeoff_maeb_mic/demora}
\end{columns}
	\vspace{0.3cm}
	\textit{NSGA--II} alcanza mejores soluciones pero requiere de un mayor tiempo de ejecución que \textit{p$\mu$MOEA/D}. %Es necesario tomar una decisión de compromiso.

	
}


% Slide - Planificador de viajes compartidos en línea ==============================================================
\section{Planificador de viajes compartidos en línea} 
\frame{\tableofcontents[currentsection]}

\frame{
	\frametitle{Planificador de viajes compartidos en línea}
	\begin{columns}[totalwidth=\textwidth]
	\column{0.5\textwidth}
	\begin{itemize}
	\item Se ingresa el origen, los destinos y la tarifa (diurna/nocturna).
	\end{itemize}
	\centering
	\includegraphics[width=.6\textwidth]{mapa.png}
	\begin{itemize}\pause
	\item Se ejecuta el AE y se muestra la planificación calculada.
	\end{itemize}
	\only<1>{\mbox{\phantom{\includegraphics[width=.6\textwidth]{res_total.png}}}}%
	\includegraphics<2->[width=.6\textwidth]{res_total.png}
	\column{0.5\textwidth}\pause
	\centering
	\only<1-2>{\mbox{\phantom{\includegraphics[width=.6\textwidth]{arquitectura.png}}}}%
	\includegraphics<3->[width=.6\textwidth]{arquitectura.png}
	\begin{itemize}
	\item Servidor implementado en Ruby on Rails siguiendo MVC.
	\item Las aplicaciones móviles consumen la API del servidor.
	\item Aplicaciones móviles: desarrollo híbrido vs. desarrollo nativo.
	\end{itemize}
	\end{columns}
}


% Slide - Conclusiones y trabajo futuro ==============================================================
\section{Conclusiones y trabajo futuro} 
\frame{\tableofcontents[currentsection]}

\frame{
	\frametitle{Conclusiones}
	\begin{itemize}
	\item Se relevó la literatura relacionada (CPP, DARP, TPP) y se presentaron \textbf{dos} variantes del problema.\pause
	\item Se implementaron \textbf{cuatro} AE: dos para cada variante del problema.\pause
	\item El análisis experimental se realizó sobre \textbf{instancias realistas}.\pause
	\item Los AE implementados fueron comparados contra algoritmos ávidos.\pause
	\item Variante monoobjetivo: mejoras en costo de hasta \alert{35.9\%} (seqEA) y \alert{41.0\%} (p$\mu$EA) sobre algoritmo ávido.\pause
	\item Variante multiobjetivo: mejoras de hasta \alert{72.8\%} y \alert{101.2\%} (p$\mu$MOEA/D); \alert{75.1\%} y \alert{105.2\%} (NSGA--II) en costo y demora sobre algoritmos ávidos.\pause
	\item Planificador de viajes compartidos en taxis, disponible públicamente en \url{www.mepaseaste.uy}.\pause
	\item \textbf{Cuatro} artículos en conferencias internacionales.\pause
	\item \textbf{Primer premio} del jurado en ``Ingeniería deMuestra 2014'' .
	\end{itemize}
}


\frame{
	\frametitle{Trabajo futuro}
	\begin{block}{Mejoras en los AE}
	\begin{itemize}
		\item Implementar NSGA--II con subpoblaciones distribuidas.\pause
		\item Incorporar \textbf{datos realistas del tráfico} para considerar rutas alternativas.\pause
		\item Incorporar datos de la \textbf{disponibilidad de los taxis} en tiempo real.\pause
	\end{itemize}
	\end{block}
	\begin{block}{Mejoras en el planificador de viajes compartidos}
	\begin{itemize}
		\item Mejorar la experiencia de los usuarios.\pause
		\item Desarrollar versiones para Android y Windows Phone.\pause
		\item Soportar la variante multiobjetivo.\pause
	\end{itemize}
	\end{block}
	\begin{block}{Problemas relacionados}
	\begin{itemize}
	\item Estudiar otras variantes del problema (many--to--one, many--to--many).\pause
	\item Estudiar la aplicabilidad de los AE a otros escenarios.
	\end{itemize}
	\end{block}
}

\frame{
	\frametitle{Fin}
	\centering
	{\huge Gracias}

	\vspace{3cm}
	Sitio web del proyecto: \url{www.fing.edu.uy/inco/grupos/cecal/hpc/AG-Taxi/}
	
}

\end{document}\grid