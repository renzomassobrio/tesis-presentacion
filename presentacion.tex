\documentclass{beamer}
%\usetheme{Berlin}
%\usetheme{Ilmenau}
%\usetheme{Dresden}
%\usetheme{Berkeley}
%\usetheme{Bergen}
%\usetheme{Boadilla}
%\usetheme{Copenhagen}
%\usetheme{Hannover}
%\usetheme{Luebeck}
%\usetheme{AnnArbor}
%\usetheme{Darmstadt}
%\usetheme{Frankfurt}
\usetheme{Madrid}%azulito-li;la
%\usetheme{Warsaw}%int
%\usetheme{Antibes}
%\usetheme{CambridgeUS}%rojo-gris
%\usetheme{Malmoe}
%\usetheme{PaloAlto}

\usepackage{graphicx}
\usepackage[utf8]{inputenc}
\usepackage[spanish]{babel}
\usepackage{url}
%\usepackage{beamerthemeshadow}
\usepackage{caption}

\hyphenation{}


%Arreglos en el footnote
\makeatletter
\setbeamertemplate{footline}
{
  \leavevmode%
  \hbox{%
  \begin{beamercolorbox}[wd=.333333\paperwidth,ht=2.25ex,dp=1ex,center]{author in head/foot}%
    \usebeamerfont{author in head/foot}\insertshortauthor
  \end{beamercolorbox}%
  \begin{beamercolorbox}[wd=.58\paperwidth,ht=2.25ex,dp=1ex,center]{title in head/foot}%
    \usebeamerfont{title in head/foot}\inserttitle
  \end{beamercolorbox}%
  \begin{beamercolorbox}[wd=.1\paperwidth,ht=2.25ex,dp=1ex,center]{date in head/foot}%
    \usebeamerfont{date in head/foot}\insertframenumber{} / \inserttotalframenumber\hspace*{2ex} 
  \end{beamercolorbox}}%
  \vskip0pt%
}
\makeatother

%Apagar el menu de presentacion
\setbeamertemplate{navigation symbols}{}%remove navigation symbols

%Bloques con sombras


\begin{document}
\title{Optimización de viajes compartidos en taxis utilizando algoritmos evolutivos}
\author[G. Fagúndez de los Reyes \and R. Massobrio]{Gabriel Fagúndez de los Reyes \and Renzo Massobrio} 
\institute[]{Facultad de Ingeniería,\\
Universidad de la República,\\
Montevideo, Uruguay}

\pgfdeclareimage[height=1.7cm]{university-logo}{logo}
\titlegraphic{\pgfuseimage{university-logo}}
\date{}

\frame{\frametitle{}\titlepage} 
\frame{\frametitle{Contenido}\tableofcontents} 

% Pausas transparentes
\setbeamercovered{transparent}

% Slide 1 ==============================================================

\section{Introducción}
\frame{\tableofcontents[currentsection]}

\frame{
	\frametitle{Motivación} 
	\begin{block}{Car pooling}
		\begin{itemize}
			\item Beneficios en el plano ecológico y económico, individuales y colectivos.
			\item Diferentes iniciativas para atender el interés del público: carriles exclusivos, campañas para compartir los viajes al trabajo y aplicaciones para encontrar compañeros de viaje.
		\end{itemize}
	\end{block}
	
	\pause %Pausa entre bloques
	
	\begin{block}{Taxi pooling}
		\begin{itemize}
			\item Los taxis son un medio de transporte rápido y confiable, especialmente en ciudades donde el transporte público es poco eficiente.
			\item Los taxis raramente viajan a capacidad completa, impactando en la congestión del tráfico y en la contaminación de las ciudades.
			\item Tarifas altas desalientan a los usuarios.
			\item \alert{15\%} de los accidentes fatales en Uruguay involucran a un conductor alcoholizado (UNASEV).
	\end{itemize}
	\end{block}
}


% Slide 2 ==============================================================

\section{Definición del problema} 
\frame{\tableofcontents[currentsection]}

\frame{
	\frametitle{Descripción del problema}
	\begin{block}{Realidad estudiada}
Un grupo de personas ubicadas en un \textbf{mismo lugar de origen}, desean viajar hacia \textbf{diferentes destinos} utilizando taxis de forma compartida.

Se busca determinar la cantidad de taxis, la asignación de pasajeros y las rutas a seguir, de forma de \alert{minimizar el costo total del grupo de pasajeros}.
	\end{block}
\pause
	\begin{block}{Consideraciones}
		\begin{itemize}
			\item Cada taxi puede trasladar a un número limitado de pasajeros.\pause
			\item El número máximo de taxis para $N$ pasajeros es $N$, en el caso particular de que cada pasajero viaje en un vehículo separado.\pause
			\item El costo de un taxi está dado por la suma del \textbf{costo inicial} (“bajada de bandera”) más el \textbf{costo determinado por la distancia} recorrida desde el origen hasta el destino final, pasando por cada uno de los destinos intermedios.\pause
			\item No se consideran otros posibles costos (e.g. esperas, propinas, peajes).
		\end{itemize}		
	\end{block}
}	


% Slide 3 ==============================================================					
\frame{
	\frametitle{Descripción del problema}
	\begin{block}{Formulación matemática}
		Dados:
		\begin{itemize}
			\item $P = \{p_1,p_2,\ldots,p_N\}$ conjunto de pasajeros que parten de $O$ y se trasladan a $D = \{d_1,d_2,\ldots,d_N\}$. \pause
			\item Una función $dest: P \rightarrow D$ que establece el destino de cada uno de los pasajeros.\pause
			\item Un conjunto de taxis $T = \{t_1,t_2,\ldots,t_M\}$; con $M \leq N$; y una función $C:T\rightarrow\{0,1,\ldots,C_{MAX}\}$ que indica la cantidad de pasajeros que viajan cada taxi. $C_{MAX}$ es la capacidad máxima permitida en un mismo taxi.\pause
			\item una constante $B$ que indica el costo inicial del taxi.\pause
			\item una función de distancia, $dist: \lbrace \lbrace O \rbrace \cup D \rbrace \times D \rightarrow \mathbb{R}^+_0$.\pause
			\item una función de costo $cost: \mathbb{R}^+_0 \rightarrow \mathbb{R}^+_0$.
		\end{itemize}
	\end{block}
}	


% Slide 4 ==============================================================					
\frame{
	\frametitle{Descripción del problema}
	\begin{block}{Formulación matemática}
		El problema consiste en hallar una planificación, es decir una función $f:P \rightarrow T \times \lbrace 1, \ldots,C_{MAX} \rbrace$ para transportar los $N$ pasajeros en $K$ taxis ($K \leq N$) que determine la asignación de pasajeros a taxis y el orden en que serán trasladados a los respectivos destinos, minimizando la función de costo total ($CT$) de la asignación, dada en la Ecuación~\ref{Eq:total_cost_mono}, donde $f^{-1}(t_i,j)$ indica el pasajero asignado a la posición $j$ (orden de traslado) en el taxi $t_{i}$ y $dest\big(f^{-1}(t_{i},0)\big)=O ,\;  \forall \: t_{i}$.

		\begin{equation}
			\label{Eq:total_cost_mono}
			CT  =  \sum\limits_{t_{i}, C(t_{i})\neq0} \Bigg[B+\sum\limits_{j=1}^{C(t_{i})}cost\bigg(dist \underbrace{\Big(dest\big(f^{-1}(t_{i},j-1)\big),des	\big(f^{-1}(t_{i},j)\big)\Big)}_{\text{destinos consecutivos en el recorrido del taxi } t_i}\bigg)\Bigg]
		\end{equation}
	\end{block}
}


% Slide 5 ==============================================================

\section{Algoritmos Evolutivos} 
\frame{\tableofcontents[currentsection]}

\frame{
	\frametitle{Algoritmos Evolutivos}

}


% Slide 6 ==============================================================

\section{Trabajo relacionado} 
\frame{\tableofcontents[currentsection]}

\frame{
	\frametitle{Trabajo relacionado}

}


% Slide 7 ==============================================================

\section{Implementación} 
\frame{\tableofcontents[currentsection]}

\frame{
	\frametitle{Implementación}

}




%%%% ================ Slide 8 - Evaluación Experimental ================ %%%%
% Slide 8.1 ==============================================================

\section{Evaluación experimental} 
\frame{\tableofcontents[currentsection]}

\frame{
	\frametitle{Evaluación experimental}
	
	\begin{block}{Generación de puntos realistas en el mapa}
		Para la generación de instancias realistas del problema se utilizó una herramienta presentada en el trabajo relacionado de Ma et al. llamada Generador de Pedidos de Taxis (\textit{Taxi Query Generator, TQG}).
	\end{block}

}

% Slide 8.2 ==============================================================
\frame{
	\frametitle{Evaluación experimental}

}

% Slide 8.3 ==============================================================
\frame{
	\frametitle{Evaluación experimental}

}

% Slide 8.4 ==============================================================
\frame{
	\frametitle{Evaluación experimental}

}

% Slide 8.5 ==============================================================
\frame{
	\frametitle{Evaluación experimental}

}

% Slide 8.6 ==============================================================
\frame{
	\frametitle{Evaluación experimental}

}

% Slide 8.7 ==============================================================
\frame{
	\frametitle{Evaluación experimental}

}

% Slide 8.8 ==============================================================
\frame{
	\frametitle{Evaluación experimental}

}


% Slide 9 ==============================================================

\section{Planificador de viajes compartidos en línea} 
\frame{\tableofcontents[currentsection]}

\frame{
	\frametitle{Planificador de viajes compartidos en línea}

}


% Slide 9 ==============================================================

\section{Conclusiones y trabajo futuro} 
\frame{\tableofcontents[currentsection]}

\frame{
	\frametitle{Conclusiones y trabajo futuro}

}

\end{document}