\documentclass{beamer}
%\usetheme{Berlin}
%\usetheme{Ilmenau}
%\usetheme{Dresden}
%\usetheme{Berkeley}
%\usetheme{Bergen}
%\usetheme{Boadilla}
%\usetheme{Copenhagen}
%\usetheme{Hannover}
%\usetheme{Luebeck}
%\usetheme{AnnArbor}
%\usetheme{Darmstadt}
%\usetheme{Frankfurt}
\usetheme{Madrid}%azulito-li;la
%\usetheme{Warsaw}%int
%\usetheme{Antibes}
%\usetheme{CambridgeUS}%rojo-gris
%\usetheme{Malmoe}
%\usetheme{PaloAlto}

\usepackage{graphicx}
\usepackage[utf8]{inputenc}
\usepackage[spanish]{babel}
\usepackage{url}
%\usepackage{beamerthemeshadow}
\usepackage{caption}


\hyphenation{}


%Arreglos en el footnote
\makeatletter
\setbeamertemplate{footline}
{
  \leavevmode%
  \hbox{%
  \begin{beamercolorbox}[wd=.333333\paperwidth,ht=2.25ex,dp=1ex,center]{author in head/foot}%
    \usebeamerfont{author in head/foot}\insertshortauthor
  \end{beamercolorbox}%
  \begin{beamercolorbox}[wd=.58\paperwidth,ht=2.25ex,dp=1ex,center]{title in head/foot}%
    \usebeamerfont{title in head/foot}\inserttitle
  \end{beamercolorbox}%
  \begin{beamercolorbox}[wd=.1\paperwidth,ht=2.25ex,dp=1ex,center]{date in head/foot}%
    \usebeamerfont{date in head/foot}\insertframenumber{} / \inserttotalframenumber\hspace*{2ex} 
  \end{beamercolorbox}}%
  \vskip0pt%
}
\makeatother

%Apagar el menu de presentacion
\setbeamertemplate{navigation symbols}{}

\begin{document}
\title{Optimización de viajes compartidos en taxis utilizando algoritmos evolutivos}
\author[G. Fagúndez de los Reyes \and R. Massobrio]{Gabriel Fagúndez de los Reyes \and Renzo Massobrio} 
\institute[]{Facultad de Ingeniería,\\
Universidad de la República,\\
Montevideo, Uruguay}

\pgfdeclareimage[height=1.7cm]{university-logo}{logo}
\titlegraphic{\pgfuseimage{university-logo}}
\date{}

\frame{\frametitle{}\titlepage} 
\frame{\frametitle{Contenido}\tableofcontents} 

% Pausas transparentes
\setbeamercovered{transparent}




% Slide - Introducción ==============================================================
\section{Introducción}
\frame{\tableofcontents[currentsection]}

\frame{
	\frametitle{Motivación} 
	\begin{block}{Car pooling}
		\begin{itemize}
			\item Beneficios en el plano ecológico y económico, individuales y colectivos.
			\item Iniciativas para atender el interés del público: carriles exclusivos, campañas para compartir los viajes al trabajo y aplicaciones para encontrar compañeros de viaje.
		\end{itemize}
	\end{block}
	
	\pause 
	
	\begin{block}{Taxi pooling}
		\begin{itemize}
			\item Los taxis son un medio de transporte rápido y confiable, especialmente en ciudades donde el transporte público es poco eficiente.
			\item Los taxis raramente viajan a capacidad completa, impactando en la congestión del tráfico y en la contaminación de las ciudades.
			\item Tarifas altas desalientan a los usuarios.
			\item \alert{15\%} de los accidentes fatales en Uruguay involucran a un conductor alcoholizado (UNASEV).
		\end{itemize}
	\end{block}
}




% Slide - Definición del problema ==============================================================
\section{Definición del problema} 
\frame{\tableofcontents[currentsection]}

\frame{
	\frametitle{Descripción del problema}
	\begin{block}{Problema de viajes compartidos en taxis (PVCT)}
		Un grupo de personas ubicadas en un \textbf{mismo lugar de origen}, desean viajar hacia \textbf{diferentes destinos} utilizando taxis de forma compartida.

		Se busca determinar la cantidad de taxis, la asignación de pasajeros y las rutas a seguir, de forma de \alert{minimizar el costo total del grupo de pasajeros}.
	\end{block}
	
	\pause
	
	\begin{block}{Consideraciones}
		\begin{itemize}
			\item Cada taxi puede trasladar a un número limitado de pasajeros.\pause
			\item el número máximo de taxis para $N$ pasajeros es $N$.\pause
			\item Costo de un taxi = \textbf{costo inicial} (“bajada de bandera”) $+$ \textbf{costo determinado por la distancia} recorrida desde el origen hasta el destino final, pasando por cada uno de los destinos intermedios.\pause
			\item No se consideran otros posibles costos (e.g. esperas, propinas, peajes).
		\end{itemize}		
	\end{block}
}

\frame{
	\frametitle{Formulación del problema}
	\begin{itemize}
		\item un conjunto de pasajeros $P = \{p_1,p_2,\ldots,p_N\}$ que viajan desde un origen común $O$ a un conjunto de destinos $D = \{d_1,d_2,\ldots,d_N\}$.\pause
		\item un conjunto de taxis $T = \{t_1,t_2,\ldots,t_M\}$; con $M \leq N$; y una función $C:T\rightarrow\{0,1,\ldots,C_{MAX}\}$ que indica la cantidad de pasajeros en un taxi. $C_{MAX}$ es la capacidad máxima permitida en un mismo taxi.\pause
		\item una constante $B$ indica el costo inicial del taxi (``bajada de bandera'').\pause
		\item una función de distancia, $dist: \lbrace \lbrace O \rbrace \cup D \rbrace \times D \rightarrow \mathbb{R}^+_0$.\pause
		\item una función de costo asociado a la distancia recorrida por cada taxi, $cost: \mathbb{R}^+_0 \rightarrow \mathbb{R}^+_0$.\pause
	\end{itemize}

	Se desea hallar la planificación $f:P \rightarrow T \times \lbrace 1, \ldots,C_{MAX} \rbrace$
	que \alert{minimice la función de costo total ($CT$)}.
	\begin{equation*}
			CT  =  \sum\limits_{t_{i}, C(t_{i})\neq0} \Bigg[B+\sum\limits_{j=1}^{C(t_{i})}cost\bigg(dist \underbrace{\Big(dest\big(f^{-1}(t_{i},j-1)\big),dest\big(f^{-1}(t_{i},j)\big)\Big)}_{\text{destinos consecutivos en el recorrido del taxi } t_i}\bigg)\Bigg]
	\end{equation*}
}

\frame{
	\frametitle{Variante multiobjetivo del PVCT}
	\begin{block}{Motivación}
		La decisión de un usuario puede estar condicionada a la demora que debe experimentar por compartir su viaje. 
		%Por tal motivo, es de interés estudiar 
		La variante multiobjetivo del problema minimiza simultáneamente el \alert{costo del grupo} de usuarios y la \alert{demora percibida} por cada uno de ellos.
	\end{block}
	
	\pause
	
	\begin{block}{Descripción}
		Se agrega un \textbf{``nivel de apuro''} asociado a cada pasajero, que denota la demora que está dispuesto a tolerar un usuario por compartir su viaje.
	
		Se contemplan \textbf{vehículos de distintas capacidades}, aportando mayor realismo a la formulación.
	\end{block}
}

\frame{
	\frametitle{Variante multiobjetivo del PVCT: formulación matemática}
	Se busca minimizar simultáneamente el \alert{costo total} y la \alert{demora total}.
	\begin{equation*}
			CT  =  \sum\limits_{t_{i}, C(t_{i})\neq0} \Bigg[B+\sum\limits_{j=1}^{C(t_{i})}cost\bigg(dist\overbrace{\Big(dest\big(f^{-1}(t_{i},j-1)\big),dest\big(f^{-1}(t_{i},j)\big)\Big)}^{\text{destinos consecutivos en el recorrido del taxi } t_i}\bigg)\Bigg]
	\end{equation*}
	\begin{equation*}
		\begin{split}
			DT = \sum\limits_{t_{i}} \Bigg[\sum\limits_{j=1}^{C(t_{i})}\bigg[ &  \overbrace{\sum\limits_{h=1}^{j} time\Big(dest\big(f^{-1}(t_{i},h-1)\big), dest\big(f^{-1}(t_{i},h)\big)\Big)}^{\text{tiempo efectivo de traslado del pasajero en la posición } j \text{ del taxi } t_i} \\
			& - \underbrace{tol\big(f^{-1}(t_{i},j)\big) + time\Big(O, dest\big(f^{-1}(t_{i},j))\Big)}_{\text{tiempo tolerado por el pasajero en la posición } j \text{ del taxi } t_i}\bigg]\Bigg]
		\end{split}	
	\end{equation*}

	\begin{itemize}
		\item $time: \lbrace \lbrace O \rbrace \cup D \rbrace \times D \rightarrow \mathbb{R}^+_0$ indica el tiempo de recorrido.	
		\item $tol: P \rightarrow \mathbb{R}^+_0$  indica el tiempo adicional tolerado por cada pasajero.
	\end{itemize}
}

\frame{
	\frametitle{Complejidad del PVCT}
	\begin{block}{Complejidad}
	El PVCT tiene varios puntos en común con dos conocidos problemas: el \textit{Car Pooling Problem (CPP)} y el \textit{Vehicle Routing Problem (VRP)}.

	Baldacci et al. (2004) estudiaron una variante del \textit{CPP} donde trabajadores desean compartir vehículos hacia y desde el lugar de trabajo. 
	
	Esta variante es un caso particular del \textit{VRP} con demanda unitaria, el cual es $\mathcal{NP}$--difícil [Letcheford et al. (2002)].
	\end{block}
	\pause
	\begin{block}{Estrategias de resolución}
	Cuando se utilizan instancias de tamaños realistas, los algoritmos exactos tradicionales no resultan útiles para una planificación eficiente.

	\alert{Heurísticas} y \alert{metaheurísticas} permiten calcular soluciones de calidad aceptable en tiempos razonables.
	\end{block}

}




% Slide - Algoritmos evolutivos ==============================================================
\section{Algoritmos evolutivos} 
\frame{\tableofcontents[currentsection]}

\frame{
	\frametitle{Algoritmos evolutivos}
	\begin{block}{Definición}
		Los \textit{algoritmos evolutivos} (AE) son técnicas estocásticas que emulan el proceso de evolución natural de las especies para resolver problemas de optimización, búsqueda y aprendizaje.
	\end{block}
	\pause
	\begin{block}{Modelos paralelos en AE}
		Las implementaciones paralelas permiten mejorar el desempe\~no de los AE. 
		Modelo de subpoblaciones distribuidas: se divide la poblaci\'on en islas que ejecutan un AE secuencial e intercambian individuos mediante \textbf{migración}.
	\end{block}
	\pause
	\begin{block}{
		El algoritmo evolutivo paralelo con micro--población (\textit{p$\mu$EA})}
		\begin{columns}
			\column{0.6\textwidth}
				Los AE con subpoblaciones distribuidas suelen perder diversidad. %, convergiendo a soluciones sub-óptimas del problema.
				\textit{p$\mu$EA} hace uso de poblaciones pequeñas e incluye un operador específico de diversidad.% para mitigar este problema.
			\column{0.3\textwidth}
				\includegraphics[width=.9\textwidth]{migracion.png}
		\end{columns}
	\end{block}
}

\frame{
	\frametitle{AE para optimización multiobjetivo (MOEA)}
	\begin{block}{Propósitos en MOEA}
		\begin{columns}
			\column{0.6\textwidth}
			Acercarse al frente de Pareto del problema (\textbf{convergencia}) y muestrear adecuadamente el frente de soluciones (\textbf{diversidad}).
			\column{0.3\textwidth}
				\includegraphics[width=.8\textwidth]{paretto.png}
		\end{columns}
	\end{block}
	\pause
	\begin{block}{\textit{p$\mu$MOEA/D}}
		\begin{columns}
			\column{0.6\textwidth}
			\begin{itemize}
			\item micro--poblaciones distribuidas
			\item aplica un esquema de agregación lineal para los objetivos
			\end{itemize}
			\column{0.3\textwidth}
				\includegraphics[width=.8\textwidth]{pmuMOEAD.png}
		\end{columns}
	\end{block}
	\pause
	\begin{block}{\textit{NSGA--II}}
			\begin{itemize}
			\item utiliza un ordenamiento no--dominado y elitista
			\item utiliza \textit{crowding} para preservar la diversidad de la población
			\end{itemize}
	\end{block}



}




% Slide - Trabajo relacionado ==============================================================
\section{Trabajo relacionado} 
\frame{\tableofcontents[currentsection]}

\frame{
	\frametitle{Trabajo relacionado}

}




% Slide - Implementación ==============================================================
\section{Implementación} 
\frame{\tableofcontents[currentsection]}

\frame{
	\frametitle{Implementación}

}




% Slide - Evaluación Experimental ==============================================================
\section{Evaluación experimental} 
\frame{\tableofcontents[currentsection]}

\frame{
	\frametitle{Evaluación experimental}
	
	\begin{block}{Generación de puntos realistas en el mapa}
		Para la generación de instancias realistas del problema se utilizó una herramienta presentada en el trabajo relacionado de Ma et al. llamada Generador de Pedidos de Taxis (\textit{Taxi Query Generator, TQG}).
	\end{block}

}

\frame{	
	\frametitle{Algoritmo ávido}
	Bla bla bla...
	\includegraphics[width=1.\linewidth]<1>{greedy_costo_1}
	\includegraphics[width=1.\linewidth]<2>{greedy_costo_2}
	\includegraphics[width=1.\linewidth]<3>{greedy_costo_3}
	\includegraphics[width=1.\linewidth]<4>{greedy_costo_4}
	\includegraphics[width=1.\linewidth]<5>{greedy_costo_5}
	\includegraphics[width=1.\linewidth]<6>{greedy_costo_6}
	\includegraphics[width=1.\linewidth]<7>{greedy_costo_7}
	\includegraphics[width=1.\linewidth]<8>{greedy_costo_8}
}

\frame{
	\frametitle{Evaluación experimental}

}

\frame{
	\frametitle{Evaluación experimental}

}

\frame{
	\frametitle{Evaluación experimental}

}

\frame{
	\frametitle{Evaluación experimental}

}

\frame{
	\frametitle{Evaluación experimental}

}

\frame{
	\frametitle{Evaluación experimental}

}

\frame{
	\frametitle{Evaluación experimental}

}


% Slide - Planificador de viajes compartidos en línea ==============================================================
\section{Planificador de viajes compartidos en línea} 
\frame{\tableofcontents[currentsection]}

\frame{
	\frametitle{Planificador de viajes compartidos en línea}

}


% Slide - Conclusiones y trabajo futuro ==============================================================
\section{Conclusiones y trabajo futuro} 
\frame{\tableofcontents[currentsection]}

\frame{
	\frametitle{Conclusiones y trabajo futuro}

}

\end{document}\grid